\documentclass[a4paper]{article}
\usepackage[english]{babel}
\usepackage{graphicx}
\usepackage[utf8]{inputenc}
\usepackage[T1]{fontenc}
\usepackage[us]{datetime}
\usepackage{multirow}

\usepackage{amsmath}
\usepackage{amsthm}
\usepackage{tocloft}
\usepackage[nottoc]{tocbibind}
\usepackage{amssymb}
\usepackage{caption}
\usepackage{float}
\usepackage{bbm}
\usepackage{theoremref}

\begin{document}

\begin{titlepage}
\begin{center}

\large \textbf{Semesterthesis}

\vspace{1cm}
\hrule
\vspace{0.4cm}
\huge \textbf{Implementation of multi-asset spread option pricing methods}\\
\vspace{0.4cm}
\hrule
\vspace{0.4cm}
\textsc{\large ETH Z\"urich}\\
\vspace{0.4cm}
\large {November 2015}

\vspace{6cm}




\vspace{4cm}
\large \emph{Author:}\\
\large Daniel W\"alchli\\
\large wadaniel@student.ethz.ch\\
\vspace{1cm}
\noindent
\begin{minipage}{0.4\textwidth}
\begin{flushleft} \normalsize
\emph{Examiner:}\\
Prof. Dr. Erich \textsc{Farkas}\\
PLE-H05\\
Plattenstrasse 22\\
8032 Z\"urich\\
farkas@math.ethz.ch
\end{flushleft}
\end{minipage}%
\begin{minipage}{0.4\textwidth}
\begin{flushright} \normalsize
\emph{Supervisor:} \\
Mrs. Fulvia \textsc{Fringuellotti}\\
PLE-G07\\
Plattenstrasse 22\\
8032 Z\"urich\\
fulvia.fringuellotti@bf.uzh.ch
\end{flushright}
\end{minipage}

\end{center}
\end{titlepage}

\pagenumbering{gobble}
\section*{Abstract}
Insert Abstract

\newpage
\pagenumbering{roman}
\setcounter{page}{1}
\renewcommand{\cftsecleader}{\cftdotfill{\cftdotsep}}
\tableofcontents

\newpage
\pagenumbering{arabic}
\setcounter{page}{1}
\section{Introduction}
A hybrid European basket-spread option is a financial derivative, whose
maturity is given by the difference (the so called spread) between two 
baskets of aggregated and weighted underlying asset prices. In mathematical terms it's pay-off is given by 
the formula:
\begin{equation}
P(S,T) = (\sum_{i=0}^M w_iS_i(T) - \sum_{j=M+1}^{M+N} w_jS_j(T) - K)^+,
\end{equation}
with $(x)^+=max(x,0)$ and where $S_i$ is the ith underlying asset price, $w_i$ its weight and $K$ is the strike price.
Basket-spread options play an important role in hedging a portfolio of correlated long and short
positions. Especially they are very common in commodity markets,
as producers are exposed to risks arising from spreads between feedstock and end products.
Basket-spread options are traded over-the-counter and on exchanges. Since there is no
closed-form solution available for the fair price it is inescapable to have an accurate and
fast approximation method at hand. The multi-dimensionality and
hence the lack of a marginal distribution for the basket-spread makes it impossible to give an exact
analytical representation for the price (even not in the simplest framework
where the asset prices are driven by  correlated geometric Brownian motions). Numerical approaches such 
as Monte Carlo simulations or PDE methods become very slow and hence inpracticble as the number of underlyings increases. 
Therefore I closely look at two different basket-spread option pricing methods which have been 
introduced by S.Deng, M.Li, and J.Zhou [1] and by G.Deelstra, A.Petkovic and M.Vanmaele [2]. According to the authors both the second-order
boundary approximation method [1,Chap.3,Prop.5] and the hybrid moment matching method associated to the improved comonotonic upper bound (HybMMICUB)
[2,Chap.3,Prop.5] are considered to be extremely fast and accurate. Therefore I implement both methods in MATLAB
and compare their numerical performances as a function of the basket spread characteristics. 
As a comparison benchmark I estimate the true prices with Monte Carlo simulations also implemented in MATLAB.\\
The paper is organized as follows.

\newpage
\section{Second-order Boundary Approximation}
Insert text.
\label{sec:sob}

\newpage
\section{Hybrid Moment Matching associated with Improved Comontonic Upper Bound}
Insert text.
\label{sec:hybmmicup}

\newpage
\section{Results and Future Work}
\label{sec:results}
Insert Results and other text.

\newpage
\begin{thebibliography}{1}
\bibitem{sob} 
S.J. Deng, M. Li, J. Zhou.
\textit{Multi-asset Spread Option Pricing and Hedging.}
October 29, 2007. School of Industrial and Systems Engineering, Georgia Institute of Technology, Atlanta.

\bibitem{sob} 
G. Deelstra, A. Petkovic, M. Vanmaele.
\textit{Pricing and Hedging Asian Basket Spread Options.}
2010. Journal of Computational and Applied Mathematics 233 2814-2830.

\end{thebibliography}

\end{document}

