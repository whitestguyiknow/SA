\documentclass[a4paper]{article}
\usepackage[english]{babel}
\usepackage{graphicx}
\usepackage[utf8]{inputenc}
\usepackage[T1]{fontenc}
\usepackage[us]{datetime}
\usepackage{multirow}

\usepackage{amsmath}
\usepackage{amsthm}
\usepackage{tocloft}
\usepackage[nottoc]{tocbibind}
\usepackage{amssymb}
\usepackage{caption}
\usepackage{float}
\usepackage{bbm}
\usepackage{theoremref}

\begin{document}

\begin{titlepage}
\begin{center}

\large \textbf{Semesterthesis}

\vspace{1cm}
\hrule
\vspace{0.4cm}
\huge \textbf{Implementation of multi-asset spread option pricing methods}\\
\vspace{0.4cm}
\hrule
\vspace{0.4cm}
\textsc{\large ETH Z\"urich}\\
\vspace{0.4cm}
\large {November 2015}

\vspace{6cm}




\vspace{4cm}
\large \emph{Author:}\\
\large Daniel W\"alchli\\
\large wadaniel@student.ethz.ch\\
\vspace{1cm}
\noindent
\begin{minipage}{0.4\textwidth}
\begin{flushleft} \normalsize
\emph{Examiner:}\\
Prof. Dr. Erich \textsc{Farkas}\\
PLE-H05\\
Plattenstrasse 22\\
8032 Z\"urich\\
farkas@math.ethz.ch
\end{flushleft}
\end{minipage}%
\begin{minipage}{0.4\textwidth}
\begin{flushright} \normalsize
\emph{Supervisor:} \\
Mrs. Fulvia \textsc{Fringuellotti}\\
PLE-G07\\
Plattenstrasse 22\\
8032 Z\"urich\\
fulvia.fringuellotti@bf.uzh.ch
\end{flushright}
\end{minipage}

\end{center}
\end{titlepage}

\pagenumbering{gobble}
\section*{Abstract}
Insert Abstract

\newpage
\pagenumbering{roman}
\setcounter{page}{1}
\renewcommand{\cftsecleader}{\cftdotfill{\cftdotsep}}
\tableofcontents

\newpage
\pagenumbering{arabic}
\setcounter{page}{1}
\section{Introduction}
A hybrid European basket-spread option is a financial derivative, whose
maturity is given by the difference (the so called spread) between two 
baskets of aggregated and weighted underlying asset prices. In mathematical terms it's pay-off is given by 
the formula:
\begin{equation}
\label{eq:po}
P(S,T) = (\sum_{i=0}^M w_iS_i(T) - \sum_{j=M+1}^{M+N} w_jS_j(T) - K)^+,
\end{equation}
with $(x)^+=max(x,0)$ and where $S_i$ is the ith underlying asset price, $w_i$ its weight and $K$ is the strike price.
Basket-spread options play an important role in hedging a portfolio of correlated long and short
positions. Especially they are very common in commodity markets,
as producers are exposed to risks arising from spreads between feedstock and end products.
Basket-spread options are traded over-the-counter and on exchanges. Since there is no
closed-form solution available for the fair price it is inescapable to have an accurate and
fast approximation method at hand. The multi-dimensionality and
hence the lack of a marginal distribution for the basket-spread makes it impossible to give an exact
analytical representation for the price (even not in the simplest framework
where the asset prices are driven by  correlated geometric Brownian motions). Numerical approaches such 
as Monte Carlo simulations or PDE methods become very slow and hence inpracticble as the number of underlyings increases. 
Therefore I closely look at two different basket-spread option pricing methods which have been 
introduced by S.Deng, M.Li, and J.Zhou [1] and by G.Deelstra, A.Petkovic and M.Vanmaele [2]. According to the authors both the second-order
boundary approximation method [1,Chap.3,Prop.5] and the hybrid moment matching method associated to the improved comonotonic upper bound (HybMMICUB)
[2,Chap.3,Prop.5] are considered to be extremely fast and accurate. Therefore I implement both methods in MATLAB
and compare their numerical performances as a function of the basket spread characteristics. 
As a comparison benchmark I estimate the true prices with Monte Carlo simulations also implemented in MATLAB.\\
The paper is organized as follows.

\newpage
\section{Approximation methods}
\label{sec:am}
In the next subsection \ref{sec:ms} I show the assumptions on which both the second-order boundary method and the HybMMICUB method are based on. Followed by subsection \ref{sec:sob} and \ref{sec:hybmmicup} which both formally present the two approximation methods. These subsections will closely follow the works of S.Deng, M.Li, and J.Zhou [1] and by G.Deelstra, A.Petkovic and M.Vanmaele [2].

\subsection{Model setup}
\label{sec:ms}
Consider M+N assets $S_1(t), S_2(t), ...,$ and $S_{M+N}(t)$ each of it following a geometric Brownian motion
$$dS_i(t) = u_iS_i(t)dt + \sigma_iS_i(t)dW_i(t).$$
The correlations of the Brownian motions $W_i(t)$ is given by the matrix $\varrho=(\rho_{i,j})$. 
The payoff of the hybrid European basket-spread has already been introduced in equation \ref{eq:po}. Conditioning on the initial asset price $S_i(0), log(S_i(T))$ is jointly normally distributed with mean $\mu_i$, variance $\sigma_i^2$ and correlation matrix $\rho_{i,j}$ for i,j = 1,2,...,M+N. 
Further the interest rate $r$ is constant and the price of the hybrid European basket-spread option is given by
\begin{equation}
\label{eq:mg}
V = e^{-rT}\mathbb{E}^{\mathbb{Q}}[(\sum_{i=0}^M w_iS_i(T) - \sum_{j=M+1}^{M+N} w_jS_j(T) - K)^+],
\end{equation}
where $\mathbb{Q}$ is the risk-neutral measure under which discounted security prices are martingales.
In the GBM setting the means and variances are defined by 
$$\mu_i = E^\mathbb{Q}[log(S_i(T))] = log(S_i(0)) + (r-\frac{1}{2}{\sigma_k}^2)T$$ 
and
$${\sigma_i}^2 = Var^\mathbb{Q}[log(S_k(T))] = \sigma_i\sqrt{T}, \; i=1,2,...,M+N.$$
Because the weights $w_i$ can be incorporated in the asset price by taking $S_i(t)'=w_iS_i(t)$ and $\mu_i=log(w_i)+\mu_i, \sigma_i'=\sigma_i$ and $\rho_{i,j}'=\rho_{i,j},$ we will assume that all weights $w_i$ equal 1 without loss of generality.

\subsection{Second-order Boundary Method}
\label{sec:sob}
The second-order boundary method for two-asset spread options (SOB) has been introduced in 2006 by Deng, Li and Zhou [3] and it was extended in 2007 to the multi-asset case [1]. Beside the SOB method Deng, Li and Zhou also introduced the extended Kirk approximation in [1]. Their work is a valuable contribution because numerous methods have been existing for spread options only involving two underlyings. In [1] Deng, Li and Zhou compare their two methods with a pricing method from Carmona and Durrleman (2005) [4] which also approximates the value of a multi-asset spread option. A study* of the results shows that the SOB method is the most accurate and also the fastest. The computational edge of the SOB method lies in its closed form solution which only involves arithmetic calculations.\\
The SOD method approximates the price of a spread option with payoff
\begin{equation}
\label{eq:sobpo}
(S_0(T) - \sum_{j=1}^{N}S_j(T) - K)^+, 
\end{equation}
thus we introduce the random variables 
$$H_0(t)=\sum_{i=1}^M S_t(t)$$ 
and 
$$H_k(t)=S_{k+M}(t), \;k=1,2,...,N$$
which can be plugged into preceding payoff (equation \ref{eq:sobpo}) and replace $S_0(T)$ and $S_j(T)$. The idea is to approximate the distribution of $H_0(T)$ by the geometric averages of the corresponding  $S_i$'s to extend the SOD method to price hybrid basked-spread options. Notice that $log(H_0(T))$ is not normally distributed nor are the $log(H_i)$'s jointly normally distributed. This comes from the fact that the sum of lognormal distributed random variables is no longer lognormal distributed.\\
The mean and the variance of the newly introduced random variable $H_0(T)$ is approximated by 
$$\mu_0^H = log(\sum_{i=0}^{M}e^{\mu_i+1/2{\sigma_i}^2}) - 1/2{\sigma_0^H}^2$$ 
and 
$$\sigma_0^H=\frac{1}{M}\sqrt{\sum_{i=1}^M \sum_{j=1}^M \rho_{i,j} \sigma_i \sigma_j}$$
Next we define random variables $X$ and $Y_k$ by
$$X = \frac{log(H_0(T))-\mu_0^H}{\sigma_0^H}, Y_k = \frac{log(H_k(T))-\mu_k^H)}{\sigma_k^H}, \: k=1,2,...,N$$
whereas $\mu_k^H = \mu_k+M$ and $\sigma_k^H = \sigma_k+M$ for $k=1,2,...,N$. Following that we can approximate the variables $X$ and $Y_k$ as jointly normally distributed with mean vector 0, variance vector 1 and correlation matrix $\Sigma = (q_{i,j})$ for $i,j=0,1,...,N$, where 
$$q_{0,0}=1,$$ 
$$q_{0,k}=q_{k,0}=1,$$
$$q_{i,j}=\rho_{M+i,M+j}.$$
Rewriting the correlation matrix $\Sigma$ as a composition of a Nx1 vector $\Sigma_{10}$ and $\Sigma_{11}$ the NxN correlation matrix of the $Y_k$'s simplifies notation for later computations:
$$\Sigma = (q_{i,j}) = \begin{pmatrix} 1 & \Sigma_{10}' \\ \Sigma_{10} & \Sigma_{11} \end{pmatrix}.$$ 
Before introducing the methods for valuing hybrid basket-spread option a short analysis of the exercise boundary is necessary. At time T, the hybrid basket-spread option is in-the-money 
if $H_0(T) - \sum_{k=1}^N S_k(T)-K>0$. If $K>0$, rewriting this inequality in terms of the random variable $X$ gives
$$X>\frac{log(\sum e^{\sigma_kY_k-\mu_k+K}-\mu_0^H}{\sigma_0^H}$$
Conditioned on $Y_k=y_k$, the option is in-the-money if $X>x(y)$, where
$$x(y)\equiv\frac{log(\sum e^{\sigma_ky_k-\mu_k+K}-\mu_0^H)}{\sigma_0^H}.$$
Note that the exercise boundary $x(y)$ is a nonlinear function in the components of y but close to being linear around $y=0$.
The analytical computation of the expectation value given in equation \ref{eq:mg} is shown below:
\begin{equation}
\label{eq:val}
V= e^{-rT} \int_{\mathbb{R}^N} \int_\mathbb{R} \big(e^{\sigma_0^Hx+\mu_0^H}-\sum_{k=1}^Ne^{\sigma_ky_k+\mu_k}-K\big)^+\phi(\{x,y\};0,\Sigma) dx dy,
\end{equation}
whereas $\phi(z;m;\Sigma)$ stands for the multivariate normal density function with mean vector m and covariance matrix $\Sigma$. Note that the random variables $X$ and $Y$ in equation $\ref{eq:val}$ are approximately jointly normally distributed with density $\phi({x,y};0;\Sigma)$.\\
Proposition 1 (Pearson (1995) [5]) allows us to reduce the (N+1)-dimensional integral from above (equation \ref{eq:val}) to N+2 N-dimensional integrals.

\newtheorem{prop}{Proposition}
\begin{prop}
\label{prop:price}
Under the jointly-normal returns setup with $K \geq 0$ and det $\Sigma \neq 0$, the price of the spread option (equation \ref{eq:sobpo}) can be written as
$$V=e^{-rT+\mu_0^H+\frac{1}{2}\sigma_0^{H2}}I_0-\sum_{k=1}^Ne^{-rT+\mu_k+\frac{1}{2}\sigma_k^2}I_k-Ke^{-rT}I_{N+1}.$$
The integrals $I_i$'s are given by
$$I_0=\int_{\mathbb{R}^N}\phi(y;0,\Sigma_{11})\Phi\big(A(y+\sigma_0^H\Sigma_{10})+\sigma_0^H\sqrt{\Sigma_{x|y}}\big)dy,$$
$$I_k=\int_{\mathbb{R}^N}\phi(y;0,\Sigma_{11})\Phi\big(A(y+\sigma_k\Sigma_{11}e_k)\big)dy, \; k=1,2,...,N,$$
$$I_{N+1}=\int_{\mathbb{R}^N}\phi(y;0,\Sigma_{11})\Phi\big(A(y)\big)dy,$$
where $e_k$ is the unit column vector (0,...,0,1,0,...,0)' with 1 at the k-th position, and
$$A(y)=\frac{\mu_{x|y}-x(y)}{\sqrt{\Sigma_{x|y}}},$$
with
$$\mu_{x|y}=\Sigma_{10}'\Sigma_{11}^{-1}y, \Sigma_{x|y}=1-\Sigma_{10}'\Sigma_{11}^{-1}\Sigma_{10}.$$
\end{prop}

In proposition \ref{prop:price}, $\Phi(z)$ stands for the one-dimensional cumulative distribution function. Note that I already plugged in the approximated mean $\mu_0^H$ and variance $\sigma_0^H$ of the random variable $H_0(t)$ instead of the mean and variance of $S_0(t)$. From now one I'll be using the approximations $\mu_0^H$ and $\sigma_0^H$.\\
Recover that det $\Sigma \neq 0$ such that $\Sigma_{x|y} \neq 0$, det $\Sigma_{11} \neq 0$ and $A(y)$ is always well defined. In the GBM setting, the price $V$ reduces to the form
\begin{equation}
\label{eq:cf}
V = S_0(t)e^{-q_0T}I_0-\sum_{k=1}^NS_k(0)e^{-q_kT}I_k-Ke^{-rT}I_{N+1}.
\end{equation}
The proof of Proposition 1 can be found in [1] in the appendix. 
The formula presented above is the starting point of our next approximation. We do a second order taylor expansion of the exercise boundary $x(y)$ and the function $A(y)$ such that the $I_k$'s can be computed in closed form. Observe that $A(y)$ is also close to being linear around $y=0$. The second order approximation of $x(y)$ and $A(y)$ is given by Proposition \ref{prop:approx}:

\begin{prop}
\label{prop:approx}
The exercise boundary x(y) can be approximated to second order in y as
$$x(y)\approx x(0)+\triangledown x|_0'y+\frac{1}{2}y'\triangledown^2x|_0y,$$
where
$$x(0)=\frac{log(R+K)-\mu_0^H}{\sigma_0^H},$$
$$(\triangledown x|_0)_k=\frac{e^{\mu_k}\sigma_k}{\sigma_0^H(R+K)},k=1,2,...,N,$$
$$(\triangledown^2 x|_0)_{i,j}=\frac{e^{\mu_i+\mu_j}\sigma_i\sigma_j}{\sigma_0^H(R+K)^2}+\delta_{i,j}\frac{v_j^2e^\mu_j}{\sigma_0^H(R+K)},i,j=1,2,...,N$$
with $\delta_{i,j}$ being the Kronecker delta function, and
$$R=\sum_{k=1}^Ne^{\mu_k}.$$
Accordingly, the function A(y) can be approximated as
$$A(y)=\frac{\mu_{x|y}-x(y)}{\sqrt{\Sigma_{x|y}}}\approx c+d'y+y'Ey,$$
where
$$c=-\frac{log(R+K)-\mu_0^H}{v_0^H\sqrt{\Sigma_{x|y}}},$$
$$d=\frac{1}{\sqrt{\Sigma_{x|y}}}(\Sigma_{11}^{-1}\Sigma_{10}-\triangledown x|_0),$$
$$E=-\frac{1}{2\sqrt{\Sigma_{x|y}}}(\triangledown^2 x|_0).$$
\end{prop}
To gain a closed-form approximation of the option price (equation \ref{eq:cf}) we need to further expand $\Phi(c+d'y+y'Ey)$ into three terms to second order in $y'Ey$ around $y'Ey=\epsilon$. The approximation formula of the price of a hybrid basket-spread is presented in Proposition \ref{prop:approx2}, which has been derived with the help of an identity in Li (2004) [6]. 
\begin{prop}
\label{prop:approx2}
Let $K\geq0$ and det$\Sigma \neq 0$. The spread option price V under the general jointly-normal returns setup is given by
\begin{equation}
\label{eq:cf2}
V=e^{-rT+\mu_0^H+\frac{1}{2}\sigma_0^{H2}}I_0-\sum_{k=1}^Ne^{-rT+\mu_k+\frac{1}{2}\sigma_k^2}I_k-Ke^{-rT}I_{N+1}.
\end{equation}
The integrals $I_i$'s are approximated as
$$I_i\approx J^0(c_i,d_i)+J^1(c_i,d_i)-\frac{1}{2}J^2(c_i,d_i), i=0,1,...,N+1,$$
where the scalar functions $J^i$ are defined as
\begin{equation}
\label{eq:j0}
J^0(u,v)=\Phi\Big(\frac{u}{\sqrt{1+v'v}}\Big),
\end{equation}
\begin{equation}
\label{eq:j1}
J^1(u,v)=\frac{\lambda}{\sqrt{1+v'v}}\phi\Big(\frac{u}{\sqrt{1+v'v}}\Big),
\end{equation}
\begin{equation}
\label{eq:j2}
\begin{split}
J^2(u,v)=&\frac{u}{(1+v'v)^{3/2}}\phi\Big(\frac{u}{\sqrt{1+v'v}}\Big)\Big\{\lambda^2+2tr[(PFP)^2]- \\
	&4\lambda(1+v'v)(v'P^2FP^2v)+(4u^2-8-8v'v)||PFP^2v||^2\Big\},
\end{split}
\end{equation}
with
\begin{equation}
\label{eq:p}
P=P(v)\equiv(I+vv')^{-1/2},
\end{equation}
$$\lambda=\lambda(u,v)\equiv u^2v'P^2FP^2v+tr(PFP)-tr(F),$$
where tr stands for the trace operator of a matrix. The scalars $c_i$, vectors $d_i$, and matrix $F$ are given by
$$c_0=c+tr(F)+\sigma_0^H\sqrt{\Sigma_{x|y}}+\sigma_0^H\Sigma_{10}'d+{\sigma_0^H}^2\Sigma_{10}'E\Sigma_{10},$$
$$d_0=\Sigma_{11}^{1/2}(d+2\sigma_0^HE\Sigma_{10}),$$
$$c_k=c+tr(F)+\sigma_ke_k'\Sigma_{11}d+\sigma_k^2e_k'\Sigma_{11}E\Sigma_{11}e_k,k=0,1,...,N,$$
$$d_k=\Sigma_{11}^{1/2}(d+2\sigma_kE\Sigma_{11}e_k),k=1,2,...,N,$$
$$c_{N+1}=c+tr(F),$$
$$d_{N+1}=\Sigma_{11}^{1/2}d,$$
$$F=\Sigma_{11}^{1/2}E\Sigma_{11}^{1/2},$$
with $\Sigma_{x|y}$ given in Proposition \ref{prop:price}.

\end{prop}

The computation in Proposition \ref{prop:approx2} appears to be cumbersome but most of the matrix multiplication can be bypassed which is shown in proposition 4. The proofs of propositions 3(4) and 4(5) can be found in [1] in the appendix. 

\begin{prop}
\label{prop:app2short}
With $P = P(v)$ as defined in equation \ref{eq:p}, we have
$$P=I-\theta vv', P^2=I-\psi vv',$$
where the scalars $\theta$ and $\psi$ are given by
$$\theta=\theta(v)=\frac{\sqrt{1+v'v}-1}{v'v+\sqrt{1+v'v}},\psi=\psi(v)=\frac{1}{1+v'v}.$$
Furthermore, we have
$$tr[(PFP)^2]=tr(F^2)-\psi(1+\psi)v'F^2v,$$
$$v'P^2FP^2v=\psi^2v'Fv,$$
$$||PFP^2v||^2=\psi^2[v'F^2v-\psi(v'Fv)^2],$$
$$tr(PFP)=tr(F)-\psi v'Fv.$$
Thus the scalar function $J^i$'s given in (\ref{eq:j0}-\ref{eq:j2}) can be simplified as
\begin{equation}
J^0(u,v)=\Phi(u\sqrt{v}),
\end{equation}
\begin{equation}
J^1(u,v)=\psi^{3/2}(\psi u^2/1)v'Fv\phi(u\sqrt{\psi}),
\end{equation}
\begin{equation}
\begin{split}
J^2(u,v)=&u\psi^{3/2}\phi(u\sqrt{v})\big\{2tr(F^2) - 4(1-tr(F))\psi-\psi^2)v'Fv+ \\
	& \psi^2(9+(2-3u^2)\psi-u^2(4-u^2)\psi^2)(v'Fv)^2-\\
	&2\psi(5+(1-2u^2)\psi)v'F^2v\big\}.
\end{split}
\end{equation}
\end{prop}

\newpage
\subsection{Hybrid Moment Matching associated with Improved Comontonic Upper Bound}
\label{sec:hybmmicup}
The hybrid moment matching method associated with improved comontonic upper bound method (HybMMICUP) combines two approximation techniques. First, the underlyings are split up according their sign and aggregated in two sums $\mathbb{S}_1$ and $\mathbb{S}_2$. Then both sums are moment matched with a log-normal random variable. Second, the price of the price of the resulting spread option with payoff $(\mathbb{S}_1-\mathbb{S}_2-K)^+$ is approximated with the improved comontonic upper bound method. The HybMMICUP procedure and other methods been presented by Deelstra, Petkovic and Vanmaele in 2009 [2]. The numerical results from [2] show that it is the best performing method to approximate the price of a basket spread option. The HybMMICUP method can also price options of Asian type but this case won’t be treated in my work.\\
More formally, the fair value of the hybrid European basket spread option will be rewritten as
\begin{equation}
V = e^{-rT}\mathbb{E}^\mathbb{Q}(S_1(T) - S_2(T) - K)^+, 
\end{equation} 
where $S_j$ is a log-normal random variable with mean $u_j = 2ln(m_{1j}) - 1/2ln(m_{2j})$ and variance $\sigma_j^2 = ln(m_{2j})-2ln(m_{1j}).$
$S_1$ and $S_2$ represent the weighted sums of the assets with positive and negative sign and $m_{1j}$ and $m_{2j}$ are the first and second moment of the respective sum $S_j$. There are different studies on how to approximate the moments of a sum of log-normal distributed random variables (such as [7] and [8]). The formula used to reproduce the results from [2] is as follows:


\newpage
\section{Results and Future Work}
\label{sec:results}
Insert Results and other text.

\newpage
\begin{thebibliography}{1}
\bibitem{sob} 
S.J. Deng, M. Li, J. Zhou.
\textit{Multi-asset Spread Option Pricing and Hedging.}
October 29, 2007. School of Industrial and Systems Engineering, Georgia Institute of Technology, Atlanta.

\bibitem{sob} 
G. Deelstra, A. Petkovic, M. Vanmaele.
\textit{Pricing and Hedging Asian Basket Spread Options.}
2010. Journal of Computational and Applied Mathematics 233 2814-2830.

\end{thebibliography}

\end{document}

